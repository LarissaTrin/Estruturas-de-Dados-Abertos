\chapter*{Por que este Livro?}
\addcontentsline{toc}{chapter}{Por que este Livro?}

Há uma abundância de livros que ensinam estruturas introdutórias de dados.
Alguns deles são muito bons. A maioria deles custam caro, e a grande 
maioria dos estudantes de graduação em ciência da computação 
desembolsará pelo menos algum dinheiro em um livro de estruturas 
de dados.

Vários livros de código aberto de estruturas de dados estão disponíveis on-line. Alguns são muito bons, mas a maioria deles estão ficando velhos. A maioria desses livros tornaram-se gratuitos quando seus autores e/ou editores decidiram parar de atualizá-los. A atualização desses livros geralmente não é possível, por duas razões: (1)~O copyright pertence ao autor e/ou editor, qualquer um dos quais não pode permitir. (2)~O \emph{código fonte} para esses livros muitas vezes não está disponível. Ou seja, o Word, WordPerfect, FrameMaker ou fonte \LaTeX\ para o livro não está disponível, e até mesmo a versão do software que manipula essa fonte pode não estar disponível.

O objetivo deste projeto é liberar estudantes de ciência da computação de graduação de ter que pagar por um livro introdutório de estruturas de dados.
Decidi implementar este objetivo tratando este livro como um projeto de software Open Source
\index{Open Source}. 
Os scripts do fonte em \LaTeX, fontes em \lang\ e de construção para o livro estão disponíveis para download no site do autor\footnote{\url{http://opendatastructures.org}} e também, mais importante, em uma fonte confiável de gerenciamento de código.\footnote{\url{https://github.com/patmorin/ods}}

O código-fonte no site é disponível sob uma licença Creative Commons Attribution, o que significa que qualquer pessoa é livre para \emph{compartilhar}:
\index{share} para copiar, distribuir e transmitir a obra; e para \emph{remixar}:
\index{remix} 
para adaptar o trabalho, incluindo o direito de fazer uso comercial da obra. A única condição para esses direitos é a \emph{atribuição}: você deve reconhecer que o trabalho derivado contém código e/ou texto de \url{opendatastructures.org}.

Qualquer um pode contribuir com correções usando o sistema de gerenciamento de código-fonte \texttt{git}.
\index{git@\texttt{git}}
Qualquer pessoa pode também derivar fontes do livro para desenvolver uma versão separada (por exemplo, em outra linguagem de programação).
Minha esperança é que, fazendo as coisas desta maneira, este livro continue a ser um livro de texto útil muito depois de terminar meu interesse no projeto, ou meu pulso,
(o que ocorrer primeiro).




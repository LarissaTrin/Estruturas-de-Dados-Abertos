\chapter{Partial Rebuilding}


In other chapters, we have seen structures that obtain their efficiency
through the use of \emph{global rebuilding}: Occasionally, the entire
structure is rebuilt in a ``balanced'' manner.  The best example of
this are the array-based list implementations of \chapref{X}.  In these
implementations, a new backing array is allocated (of appropriate size)
and all the elements in the old backing array arec copied into the new
backing array.

In this chapter, we consider structures that maintain their efficiency
through the use of \emph{partial rebuilding}.  In applications of this
technique, only parts of the data structure are rebuilt.

\chapter{ScapegoatTree: A binary tree with partial rebuilding}

A #ScapegoatTree# is a binary search tree that, in addition to storing
a reference #root# to the root node, store the number #n# of nodes in
the tree and another number #q# that is an overestimate of the number
of nodes in the tree.  At all times, two invariants are maintained.
The first invariant ensures that #q# is a reasonable estimate of #n#:
\[  #n# \le #q# \le 2#n# \enspace . \]
The second invariant ensures that the height of the tree is small
\[
 \mbox{height} \le \log_{3/2} q
\]



\chaplabel{scapegoat}
\chapter{Jumplist: Randomized Partial Rebuilding}
\chapter{KDTree: Multidimensional search trees}


\chaplabel{jumplists}


